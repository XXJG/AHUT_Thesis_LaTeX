
\chapter{模板的使用说明}{The usage guide of this template}

\section{背景}{background}

我非常推荐使用LaTeX来进行论文的编辑,因为它可以让你真正的进行内容的思考而不需要考虑格式的问题。目前正是编写毕业论文的时候,
我深感word给我带来的格式的麻烦,所以才有了此次模板。

最好是使用XeLaTeX进行编译,在你的Mac或者Windows上安装MacTeX或者TeXLive是非常重要的。


\section{使用说明}{Some notes}

为了正确使用该模板,请按照提示安装好可使用的TeX发行版本。因为论文内容比较多,因此采取了分文件的方式来构成该文档。
主文档AHUT.tex的位置位于Main下,正确编译后所得的pdf文件就在这里。Figure文件夹是存放图片的文件夹,该文件夹已经加入图片文件夹的位置,插入图片是无需多加路径,直接用文件名即可。
关于Setting文件夹只需要把里面的Information.tex正确填入即可。
而你需要编辑的仅有Body文件夹下的文件。

该模板是在厦门大学博士学位论文模板的基础上修改得到的,因为安徽工业大学本科论文一直未有LaTeX模板,所以定制了该模板。
由于本人水平有限,因此该模板写的并不好,但是应该勉强能够满足毕业论文的要求。同时本文也有很多错误之处,还有关于字体和行间距的设置问题,
由于时间有限并未对其进行调整,希望有能力的同学可以帮忙提出Issue,共同建设母校的LaTeX。


联系方式:
邮箱: \href{mailto:xcharles@foxmail.com}{xcharles@foxmail.com}

github项目的地址 : \href{https://github.com/XXJG/AHUT_Thesis_LaTeX}{AHUT_Thesis_LaTeX}
